\section{Introduction to Coq}

[Coq](https://coq.inria.fr/) is a system for theorem proving, called a *proof assistant*.
It includes several sub-programming languages for variable specifications of definitions and proofs.

\subsection{Curry-Howard Isomorphism}

The whole thing is based on the Curry-Howard isomorphism~\cite{curry-howard}; the basic idea is that *types are propositions, programs are proofs*.
What does that mean?

Essentially, if it is possible to construct a value of some type, $\tau$, then that type is "true", in the sense that we can produce a proof of it.
And what is the proof?
The value of type $\tau$.
This probably seems weird, because it's weird to think of the type \mintinline{coq}{Int} as being "true", but it just means that I can prove that an \mintinline{coq}{Int} exists: just pick your favorite one (mine is \mintinline{coq}{0}).

In this case, a slightly more complicated example will probably make more sense.
Let's consider implication: if $P$, then $Q$.
What does this correspond to as a type?
A function!
Proving if $P$, then $Q$ is the same as writing a function that takes a value of type $P$ and produces a value of type $Q$.
In Java we might write:

\begin{minted}{java}
public <P, Q> Q f(P p) {
    // ...
}
\end{minted}

but since this series is about Coq, I'll mostly write in Coq from now on:

\begin{minted}{coq}
Definition f : P -> Q := (* ... *).
(* alternatively *)
Definition f (p : P): Q := (* ... *).
\end{minted}

As a more concrete example of implication, take the following introduction rule in intuitionistic logic~\cite{intuit-logic}.

$$ \frac{A~~~B}{A \wedge B} $$

This rule says that if $A$ is true, and $B$ is true, then $ A \wedge B $ is true.
This is a fairly obvious definition for what "and" means, and it's a nice and simple example to demonstrate how the Curry-Howard isomorphism works.
Recalling that "types are propositions", let's consider what the proposition here is: if $A$ and $B$, then $A \wedge B$.
What would this mean in a programming lanugage?
If you have a value of type \mintinline{coq}{a}, and a value of type \mintinline{coq}{b}, then you can create a value which contains values of both type \mintinline{coq}{a} and \mintinline{coq}{b}.
That is, a *pair*, of the two values.

In Haskell, we would write this as:

\begin{minted}{haskell}
prop :: a -> b -> (a, b)
prop a b = (a, b)
\end{minted}

The first line, the type, is the proposition.
The second line, the definition of the function, is the proof: a program that shows how to go from the assumptions (a value of type \mintinline{coq}{a} exists, and a value of type \mintinline{coq}{b} exists), to the conclusion (so there is a value which contains values of both types).

Translating to Coq, this becomes:

\begin{minted}{coq}
Definition conjunc (a : A) (b : B) : A /\ B := conj a b.
\end{minted}

